%%% Local Variables: 
%%% mode: xelatex
%%% TeX-master: t
%%% End: 
 
%\documentclass[a4paper,10pt]{scrreprt}
%\documentclass[a4paper,10pt]{article}
%\documentclass[a4paper,10pt]{report}

\documentclass[unicode, 12pt, a4paper,oneside,fleqn]{article}
\usepackage[cm-default]{fontspec}
\defaultfontfeatures{Mapping=tex-text}    %% устанавливаем поведение шрифтов по умолчанию
\usepackage{polyglossia}    %% подключаем пакет многоязыкой верстки
%\setdefaultlanguage{russian}    %% установка языка по умолчанию
\setdefaultlanguage{english}    %% установка языка по умолчанию
%\setotherlanguages{english}
\setmainfont{Old Standard}      %% зададим основной шрифт документа
%\setmainfont{DejaVu Sans Mono}
\setmonofont{DejaVu Sans Mono}

\usepackage{mathtext}               % если нужны русские буквы в формулах
%\usepackage{ucs}
%\usepackage[utf8x]{inputenc}       % Кодировка входного документа;
                                   % при необходимости, вместо cp1251
                                   % можно указать cp866 (Alt-кодировка
                                   % DOS) или koi8-r.

\usepackage{textcomp}              % типографские значки

% \usepackage[T2A]{fontenc}           % Кодировка для шрифтов LH
%\usepackage{indentfirst}    % неизвестно
%\usepackage{cmap}           % неизвестно
%\usepackage[english,russian]{babel} % Включение русификации, русских и
                                    % английских стилей и переносов

%\renewcommand{\rmdefault}{ost_____} % add new font  Old_Standard
%\renewcommand{\sfdefault}{ost_____}
%\renewcommand{\ttdefault}{osti____}

\usepackage{graphics}
\usepackage{pgf}
\usepackage{wrapfig}
\usepackage{multicol}
\usepackage{multirow}
\usepackage{tabularx}
%\usepackage{fullpage}
%\usepackage{amsmath} % для спец знаков в формулах
%\usepackage{amssymb} % для спец знаков в формулах
\usepackage{topcapt} % подписи к таблицам
\usepackage{dcolumn} % выравнивание чисел
\usepackage{ulem} % подчёркивание

\hyphenpenalty=10000 % запретить переносы
%\tolerance-1 
\pretolerance10000 


\usepackage[colorlinks=true]{hyperref} % url hyperlink

\usepackage{makeidx} % индекс

%\newfontfamily\cyrillicfont{DejaVu Sans Mono}
%\newfontfamily{\cyrillicfont}{Liberation Mono}
\usepackage{verbatim} % печать неформатированного текста
%\usepackage{verbatimbox} % печать неформатированного текста в рамке
\usepackage{fancyvrb}


\usepackage{listings} % печать исходного кода
%\lstloadlanguages{lisp}
\lstset{
  language=Pascal,
  %basicstyle=\tiny, %or \small or \footnotesize etc.
  extendedchars=\true, %Чтобы русские буквы в комментариях были 
  %texcl,  %Чтобы русские буквы в комментариях были не слипшимися
  keepspaces = true, %Чтобы русские буквы в комментариях были не слипшимися это работает!!!!!!!
  escapechar=|,
  frame=single,
  commentstyle=\itshape,
  inputencoding=utf8x,
  stringstyle=\bfseries
}



% вращение 
\usepackage{lscape}     % for %\begin{landscape} ...   %\end{landscape}
\usepackage{rotating}   % for sideways and \rotatebox{-90}{}




\author{Р.В.~Приходченко}
\title{Краткое руководство по языку программирования Pascal}
\frenchspacing


\makeindex

\begin{document}

%\renewcommand\bibname{СПИСОК ЛИТЕРАТУРЫ} 
\renewcommand\refname{\centering Список литературы}

\newcolumntype{Y}{>{\centering\arraybackslash}X}
%\newcolumntype{R}{>{\raggedleft\arraybackslash}X}


\maketitle
\tableofcontents


\begin{table}[ht]
  \begin{tabular}{cc}
    \includegraphics[width=2cm]{CC_BY-SA_88x31.png} &
    \shortstack{руководство распространяется в соответствии с
      условиями\\
      \href{http://creativecommons.org/licenses/by-sa/3.0/}{Attribution-ShareAlike} \\
      (Атрибуция — С сохранением условий) CC BY-SA \\
      Копирование и распространение приветствуется.}
  \end{tabular}
\end{table}

\section*{Введение}
По моему мнению при изучении низкоуровневых (паскаль создавался как
подготовка к языку C, а язык C по мнению авторов языка C - переносимый
ассемблер) языков, чтобы написать <<Hello world!>>, не стоит
использовать <<Интегрированные среды разработки>> (IDE). Потому что,
помимо самого языка придётся изучать IDE, которые порой ещё более
запутанные, чем изучаемый язык программирования. В большинстве
текстовых редакторов есть подсветка синтаксиса и парных скобок,
автодополнение или сниппеты, автоматическое выравнивание кода, а
компилировать можно в терминале (хотя некоторые редакторы позволяют
компилировать по команде), всего этого для начала должно хватить. В
дальнейшем, скорее всего, студент не будет работать в паскале: на
третьем курсе начинают изучать язык C и к тому времени сможет сам
выбрать IDE, а изучение Lazarus-а или Delphi (паскалевские IDE)
окажется почти напрасным.

\subsection{Приблизительная последовательность действий при написании
  программы}
все действия этого параграфа происходят в терминале (консоль) кроме
пункта 4; пункты нужно выполнять последовательно
\begin{enumerate}
\item создать каталог. Это необходимо сделать только один раз перед
  началом нового проекта (программы). Каждый проект хранится в
  отдельном каталоге. Каталог проекта будет содержать исходный код
  программы, исполняемые файлы, руководство пользователя и другие
  файлы необходимые для работы программы.

  пример:

\begin{lstlisting}[language=bash]
mkdir -p 21119/petroff/proj_3_abc
\end{lstlisting}
  
  где \begin{description}
  \item $ 21119 $ -- номер группы,
  \item $ petroff $ -- фамилия,
  \item $ proj\_3\_abc $ -- название проекта.
  \end{description}
  название проекта (каталог, а также имена файлов и все имена
  переменных) необходимо выбирать основательно в данном случае
  proj\_3\_abc расшифровывается как проект программы по распечатке
  алфавита, цифра 3 обозначает номер задания.

  Возможно несколько вариантов написания сложных названий:
  \begin{enumerate}
  \item Слитное написание: proj3abc - не рекомендуется
  \item CamelCase: Proj3Abc либо proj3Abc (в некоторых языках приняты
    разные варианты написания для переменных и функций)
  \item snake-case (используя разделитель <<подчёркивание>> либо <<тире>>):
    proj\_3\_abc
  \item Венгерская нотация - использование префиксов (s - string, i -
    int, b - boolean, a - array, us - небезопасные, sf - безопасные):
    susClientName, iusSize, asfDimensions, однако компилятор в
    <<нормальных языках>> и так знает типы (в том числе и
    пользовательские) и может их проверить
  \item Смешанное: Proj\_3\_Abc или PROJ\_3\_ABC
  \end{enumerate}



%  {\tiny
%    
%    Which is better: identifier names that\_look\_like\_this or
%    identifier names thatLookLikeThis?
%    
%    It's a precedent thing. If you have a Pascal or Smalltalk
%    background, youProbablySquashNamesTogether like this. If you have
%    an Ada background, You\_Probably\_Use\_A\_Large\_Number\_Of\_Underscores
%    like this. If you have a Microsoft Windows background, you
%    probably prefer the "Hungarian" style which means you jkuidsPrefix
%    vndskaIdentifiers ncqWith ksldjfTheir nmdsadType. And then there
%    are the folks with a Unix C background, who abbr evthng n use vry
%    srt idntfr nms. (AND THE FORTRN PRGMRS LIMIT EVRYTH TO SIX
%    LETTRS.)
%    
%  }


%  CamelCase настолько читаем, что в Emacs даже есть специальный режим:
%  glasses minor mode makes ‘unreadableIdentifiersLikeThis’ readable by
%  altering the way they display.
  

  также заранее подумайте что правильнее для инструкции:
  \begin{enumerate}
  \item чай\_пей (объект\_действие) % действие=метод
  \item пей\_чай 
  \end{enumerate}



  Выбор варианта стиля:
  \begin{enumerate}
  \item работа в команде - команда выберет стиль за вас.
  \item продолжение работы над уже существующим проектом - предыдущий
    автор уже выбрал стиль за вас.
  \item ВЯзыкеПрограммированияИспользуется ОпРеДеЛёНнЫйСтИлЬ -
    язык\_программирования\_выберет стиль-за-вас.
  \item в остальных случаях выбор стиля за вами. Да.
  \end{enumerate}
  
  Выбранный вариант желательно использовать не только для каталогов,
  но и для названий файлов, а также во всей программе для функций и
  переменных и прочего.


\item перейти в каталог проекта
\begin{lstlisting}[language=bash]
cd 21119/petroff/projabc3
\end{lstlisting}

\item запустить любимый текстовый редактор, например: emacs,
  vim(gvim), mc(mcedit), gedit.

  Запускать не обязательно из терминала, можно из <<Меню программ>>:
  Системные или Разработка, а может быть Инструменты э-э-э-э нет
  всё-таки Прочее если тоже не обнаружилось то наверное пропустили в
  Системных > Простой редактор текстов > [уже в простом редакторе
  текстов] Файл > Открыть > найти и выбрать двойным щелчком 21119 >
  найти и выбрать двойным щелчком petroff > найти и выбрать двойным
  щелчком projabc3 > найти и выбрать двойным щелчком abc3.pas (хотя
  некоторые могут заметить что в терминале подобное уже было сделано
  командой cd 21119/petroff/projabc3, но терминал нам потребуется
  запускать для ручной компиляции)
  
  
  а в терминале:

\begin{lstlisting}[language=bash]
emacs abc3.pas &
\end{lstlisting}

  где \begin{description}
  \item $ abc3.pas $ -- название программы - должно быть связано с
    названием проекта и вместо abc3.pas желательно использовать
    projabc3.pas,
  \item $ \& $ -- (амперсанд) - интерпретатор (bash) не дожидается
    завершения команды, выполнение программы (emacs) происходит в
    фоновом режиме (в терминале можно вводить команды не останавливая
    emacs)
  \end{description}

\item в текстовом редакторе (например emacs) самостоятельно написать
  хорошую, правильную программу.

  Правильная программа предполагает хорошее оформление. Прочитать
  обязательно: \cite[Жиганов Е.Д.]{zed.coding.rules}.
  
  Для упрощения процесса написания программы (на примере emacs):
  \begin{enumerate}
  \item используйте табуляцию для отступов
  \item после того как запомните однотипные базовые конструкции
    (например program ... uses ... const ... var ... begin ... end.),
    включите сниппеты (snippet)
    
    например: в редакторе пишите program (и больше ничего) потом
    нажимаете <<Tab>> и появляется заготовка целого блока программы
    program ... uses ... const ... var ... begin ... end.
  \item как можно чаще сохраняйте программу (в компьютерных классах
    старые компьютеры - возможны зависания)
  \item за неделю с компьютером может случиться разное - например на
    лабораторных по эксплуатации ЭВМ будут изучать файловые системы и
    ... , поэтому в конце занятия сохраните программу на:
    \begin{enumerate}
    \item флэшку лучше в каталог с датой и версией, например:
      2014-11-15-v1.2, а перед тем как вытащить флэшку ВСЕГДА
      отмонтируйте файловую систему не зависимо от операционной
      системы
    \item интернет сервис (возможно потребуется разрешить java скрипты
      - в правом верхнем углу перечёркнутая буква S разрешить
      pastebin.com)
      
      \href{http://pastebin.com/}{http://pastebin.com/}
      
      после отправки, получите короткий код типа
      http://pastebin.com/cOcle, который аккуратно записываете
      повторяя все {\tiny маленькие} и БОЛЬШИЕ буквы и цифры
      (осторожно - в коде cOcle второй символ - цифра ноль, а третья и
      четвёртая буквы cl сливаются в букву d)
    \item интернет сервис
      
      \href{https://gist.github.com/}{https://gist.github.com/}
      
      аналогично получите код
    \end{enumerate}

  \item после того как вдоволь насохраняетесь - обязательно узнайте
    что такое <<системы контроля версий [d]vcs>> например: git.
  \item к этому моменту вы уже либо превратите emacs/vim в IDE, либо
    найдёте IDE по своему вкусу, или забросите программирование как
    большинство.
  \end{enumerate}
  
\item компиляция программы компилятором (fpc)
  
  процесс получения исполняемого файла из исходных текстов программы

\begin{lstlisting}[language=bash]
fpc abc3.pas
\end{lstlisting}

  где \begin{description}
  \item $ abc3.pas $ -- название программы.
  \end{description}


  Однако лучше использовать гламурную компиляцию; для этого нужно в
  терминале ввести команду (не забудьте написать команду в одну
  строчку, а также поменять типографские кавычки на одинарные кавычки)

%  {\scriptsize
%\begin{verbatim}
%function fpcc() { fpc "$1" 2>&1 | grep -Ei --color 'error|fatal|warning|note|'; }
%\end{verbatim}
%  }

%  \scriptsize
%    \begin{verbbox}
%      function fpcc() { fpc "$1" 2>&1 | grep -Ei --color 'error|fatal|warning|note|'; }
%    \end{verbbox}
%    \fbox{\addhbuffer[-10pt 0pt]{\theverbbox}}
  

\begin{Verbatim}[fontsize=\tiny,frame=single]
function fpcc() { fpc "$1" 2>&1 | grep -Ei --color 'error|fatal|warning|note|'; }
\end{Verbatim}

%\begin{lstlisting}[language=bash]
%function fpcc() { fpc "$1" 2>&1 | grep -Ei --color 'error|fatal|warning|note|'; }
%\end{lstlisting} 


 
и запускать
\begin{lstlisting}[language=bash]
fpcc abc3.pas
\end{lstlisting}

% %%%%%%%%%%%%%%%%%%%%%%%%%%%%%%%%%%%%%%%%%%%%%%%%%%%%%%%%%%%%%%%%%%%%%%%%%%%%%%%%%%%%%%%%%%%%%%%%%%%%%%%%%%%%%%%%%%%%%%%%%%%%%%%%%%%%%%%%%%%%%%%%%%%%%%%%%%%%%%%%%%%%%%%%%%%%%%%%%%%%%%%%%%%%%%%%%%%%%%%%%%%%%%%%%%%%%%%%%%%%%%%%%%%%%%%%%%%%%%%%%%%%%%%%%%%%%%%%%%%%%%%%%%%%%%%%%%%%%%%%%%%%%%%%%%%%%%%%%%%%%%%%%%%%%%%%%%%%%%%%%%%%%%%%%%%%%%%%%%%%%%%%%%%%%%%%%%%%%%%%%%%%%%%%%%%%%%%%%%%%%%%%%%%%%%%%%%%%%%%%%%%%%%%%%%%%%%%%%%%%%%%%%%%%%%%%%%%%%%%%%%%%%%%%%%%%%%%%%%%%%%%%%%%%%%%%%%%%%%%%%%%%%%%%%%%%%%%%%%%%%%%%%%%%%%%%%%%%%%%%%%%%%%%%%%%%%%%%%%%%%%%%%%%%%%%%%%%%%%%%%%%%%%%%%%%%%%%%%%%%%%%%%%%%%%%%%%%%%%%%%%%%%%%%%%%%%%%%%%%%%%%%%%%%%%%%%%%%%%%%%%%%%%%%%%%%%%%%%%%%%%%%%%%%%%%%%%%%%%%%%%%%%%%%%%%%%%%%%%%%%%%%%%%%%%%%%%%%%%%%%%%%%%%%%%%%%%%%%%%%%%%%%%%%%%%%%%%%%%%%%%%%%%%%%%%%%%%%%%%%%%%%%%%%%%%%%%%%%%%%%%%%%%%%%%%%%%%%%%%%%%%%%%%%%%%%%%%%%%%%%%%%%%%%%%%%%%%%%%%%%%%%%%%%%%%%%%%%%%%%%%%%%%%%%%%%%%%%%%%%%%%%%%%%%%%%%%%%%%%%%%%%%%%%%%%%%%%%%%%%%%%%%%%%%%%%%%%%%%%%%%%%%%%%%%%%%%%%%%%%%%%%%%%%%%%%%%%%%%%%%%%%%%%%%%%%%%%%%%%%%%%%%%%%%%%%%%%%%%%%%%%%%%%%%%%%%%%%%%%%%%%%%%%%%%%%%%%%%%%%%%%%%%%%%%%%%%%%%%%%%%%%%%%%%%%%%%%%%%%%%%%%%%%%%%%%%%%%%%%%%%%%%%%%%%%%%%%%%%%%%%%%%%%%%%%%%%%%%%%%%%%%%%%%%%%%%%%%%%%%%%%%%%%%%%%%%%%%%%%%%%%%%%%%%%%%%%%%%%%%%%%%%%%%%%%%%%%%%%%%%%%%%%%%%%%%%%%%%%%%%%%%%%%%%%%%%%%%%%%%%%%%%%%%%%%%%%%%%%%%%%%%%%%%%%%%%%%%%%%%%%%%%%%%%%%%%%%%%%%%%%%%%%%%%%%%%%%%%%%%%%%%%%%%%%%%%%%%%%%%%%%%%%%%%%%%%%%%%%%%%%%%%%%%%%%%%%%%%%%%%%%%%%%%%%%%%%%%%%%%%%%%%%%%%%%%%%%%%%%%%%%%%%%%%%%%%%%%%%%%
%
% http://www.linux.org.ru/forum/development/4184158
% http://creativecommons.org/licenses/
% http://legroom.net/2009/08/18/bash-shell-aliases-and-functions
%
% %%%%%%%%%%%%%%%%%%%%%%%%%%%%%%%%%%%%%%%%%%%%%%%%%%%%%%%%%%%%%%%%%%%%%%%%%%%%%%%%%%%%%%%%%%%%%%%%%%%%%%%%%%%%%%%%%%%%%%%%%%%%%%%%%%%%%%%%%%%%%%%%%%%%%%%%%%%%%%%%%%%%%%%%%%%%%%%%%%%%%%%%%%%%%%%%%%%%%%%%%%%%%%%%%%%%%%%%%%%%%%%%%%%%%%%%%%%%%%%%%%%%%%%%%%%%%%%%%%%%%%%%%%%%%%%%%%%%%%%%%%%%%%%%%%%%%%%%%%%%%%%%%%%%%%%%%%%%%%%%%%%%%%%%%%%%%%%%%%%%%%%%%%%%%%%%%%%%%%%%%%%%%%%%%%%%%%%%%%%%%%%%%%%%%%%%%%%%%%%%%%%%%%%%%%%%%%%%%%%%%%%%%%%%%%%%%%%%%%%%%%%%%%%%%%%%%%%%%%%%%%%%%%%%%%%%%%%%%%%%%%%%%%%%%%%%%%%%%%%%%%%%%%%%%%%%%%%%%%%%%%%%%%%%%%%%%%%%%%%%%%%%%%%%%%%%%%%%%%%%%%%%%%%%%%%%%%%%%%%%%%%%%%%%%%%%%%%%%%%%%%%%%%%%%%%%%%%%%%%%%%%%%%%%%%%%%%%%%%%%%%%%%%%%%%%%%%%%%%%%%%%%%%%%%%%%%%%%%%%%%%%%%%%%%%%%%%%%%%%%%%%%%%%%%%%%%%%%%%%%%%%%%%%%%%%%%%%%%%%%%%%%%%%%%%%%%%%%%%%%%%%%%%%%%%%%%%%%%%%%%%%%%%%%%%%%%%%%%%%%%%%%%%%%%%%%%%%%%%%%%%%%%%%%%%%%%%%%%%%%%%%%%%%%%%%%%%%%%%%%%%%%%%%%%%%%%%%%%%%%%%%%%%%%%%%%%%%%%%%%%%%%%%%%%%%%%%%%%%%%%%%%%%%%%%%%%%%%%%%%%%%%%%%%%%%%%%%%%%%%%%%%%%%%%%%%%%%%%%%%%%%%%%%%%%%%%%%%%%%%%%%%%%%%%%%%%%%%%%%%%%%%%%%%%%%%%%%%%%%%%%%%%%%%%%%%%%%%%%%%%%%%%%%%%%%%%%%%%%%%%%%%%%%%%%%%%%%%%%%%%%%%%%%%%%%%%%%%%%%%%%%%%%%%%%%%%%%%%%%%%%%%%%%%%%%%%%%%%%%%%%%%%%%%%%%%%%%%%%%%%%%%%%%%%%%%%%%%%%%%%%%%%%%%%%%%%%%%%%%%%%%%%%%%%%%%%%%%%%%%%%%%%%%%%%%%%%%%%%%%%%%%%%%%%%%%%%%%%%%%%%%%%%%%%%%%%%%%%%%%%%%%%%%%%%%%%%%%%%%%%%%%%%%%%%%%%%%%%%%%%%%%%%%%%%%%%%%%%%%%%%%%%%%%%%%%%%%%%%%%%%%%%%%%%%%%%%%%%%%%%%%%%%%%%%%%%%%%%%%%%%%%%%%%%%%%%%%%%%%%%%%%%%%%%%%%%%%%%%%%%%%%%%%%%%%%%%%%%%%%%%%%%%%%%%%%%%%%%%%%

или можно создать файл \verb!~/bin/fpcc.sh! с таким содержимым:

\begin{Verbatim}[fontsize=\footnotesize,frame=single]
#!/bin/sh
fpc $1 2>&1 | grep -Ei --color 'error|fatal|warning|note|'
\end{Verbatim}


тогда запускать 
\begin{lstlisting}[language=bash]
~/bin/fpcc.sh abc2.pas
\end{lstlisting}





\item внимательно прочитать сообщения компилятора; при наличии ошибок
  или предупреждений перейти к пункту 4 (о сообщениях компилятора
  см. ниже)
  
\item запустить программу\\

  
\begin{lstlisting}[language=bash]
./abc3
\end{lstlisting}

  где \begin{description}
  \item $ ./ $ -- текущий каталог,
  \item $ abc3 $ -- название исполняемого файла (без расширения <<.pas>>).
  \end{description}


  
\item если программа получилась негодной, перейти к пункту 4

\item если для демонстрации программы необходимо построить график то
  получаем текстовый файл с несколькими колонками разделёнными
  запятыми (без лишних сообщений), например так:

\newpage

\begin{lstlisting}
program abc5;
uses math;
const
   h: real = 1.0e-1;
var 
   a,b,c : real;

begin
   a:=0.0;
   b:=5.0;
   c:=a;
   repeat
      writeln(c, ', ', sin(c));
      c := c + h;
   until (c>b);
   
end.
\end{lstlisting}


запускаем с перенаправлением стандартного вывода внутрь файла:
\begin{lstlisting}[language=bash]
./abc3 > data.txt
\end{lstlisting}

в zsh, если файл data.txt уже есть, запускаем так:
\begin{lstlisting}[language=bash]
./abc3 >! data.txt
\end{lstlisting}


  
для построения графика можно воспользоваться программой R или gnuplot
(в них можно строить даже трёхмерные поверхности)

  \begin{enumerate}
  \item R
    
    запускаем в терминале R
{\begin{lstlisting}[language=R,basicstyle=\small]
gr <- read.table("data.txt", sep=",", head=FALSE)
plot(gr, type="l")
\end{lstlisting}}

  \item gnuplot
    
    запускаем в терминале gnuplot
\begin{lstlisting}[language=gnuplot]
plot "data.txt" with line
\end{lstlisting}
  \end{enumerate}
\end{enumerate}

выход <<Ctrl + d>>




\subsection{Сообщения компилятора}

Компилятор показывает сообщения об ошибках с номером строки и номером
символа в круглых скобках.\\
Например (6,4) - ошибка в строке 6, номер символа 4.\\
Однако если отсутствует \verb!;! (точка с запятой) в конце оператора
то компилятор укажет на следующую строку (пропущенную точку с запятой,
скорее всего, нужно добавить строкой выше). Если вы воспользовались
гламурной компиляцией (смотри выше) то ключевые слова будут подсвечены
цветом.
 
Если в процессе компиляции появляются сообщения со словами <<error>>
или <<fatal>>, то в программе присутствует ошибка, которую необходимо
исправить. Например ошибки синтаксиса и операции с различными типами:
{\tiny
\begin{verbatim}
abc3.pas(6,4) Fatal: Syntax error, "." expected but ";" found
abc3.pas(7,4) Error: Incompatible types: got "String" expected "Real"
abc3.pas(10) Fatal: There were 1 errors compiling module, stopping
Fatal: Compilation aborted
\end{verbatim}
}

Если в процессе компиляции появляются сообщения со словами <<warning>>
или <<note>>, то в программе присутствует недостаток, котоый
желательно исправить. Например неиспользуемая переменная и
неинициализированная переменная (объявили переменную, в неё ничего не
записали, попытались вывести её значение на экран):

{\tiny
\begin{verbatim}
abc3.pas(3,7) Note: Local variable "c" not used
abc3.pas(10,16) Warning: Variable "b" does not seem to be initialized
\end{verbatim}
}

успешно откомпилированная программа должна содержать примерно такую
строку:
\begin{verbatim}
10 lines compiled, 0.0 sec
\end{verbatim}


\section{Паскаль}

примерный вид очень простой программы


\begin{lstlisting}

program abc3;     // название программы начинается с буквы

var               // описание типов переменных
   s : string;
   a : integer;

begin             // начало программы

   a := 5;        // программа
   readln(s);     // программа
   writeln(s, a); // программа
   
end.              // конец программы - end с точкой

\end{lstlisting}





примерный вид программы посложней


\begin{lstlisting}
program abc3;     // название программы (начинается с буквы)

uses math;       // подключение модулей
                // (в данном случае для математических функций)

const             // список констант
   MAX : integer = 100;  

type
   mass : array [1..MAX] of integer;


var               // описание типов переменных
   a : integer;
   s : string;
   m : mass;

begin             // начало программы

   readln(s);     // программа
   m[2] := 7;     // программа
   m[MAX-8] := 3; // программа
   a := 5;        // программа
   writeln(s, a); // программа
   
end.              // конец программы (end с точкой)
\end{lstlisting}



\subsection{Ключевые слова}
ключевые слова не допустимо использовать для названия переменных,
констант, процедур и функций.  список ключевых слов:

absolute, and, array, asm, begin, boolean, break, case, char, const,
continue, div, do, downto, else, end, for, function, goto, if,
implementation, in, interrupt, is, label, mod, not, or, org,
otherwise, print, procedure, program, read, real, record, repeat, shl,
shr, step, string, then, to, type, unit, until, uses, var, while,
with, xor

\subsection{Комментарии}
текст заключённый между фигурными скобками - комментарии к программе
Пример:
\begin{verbatim}
{ Место для комментария
  Комментарий может занимать несколько строк }
\end{verbatim}

текст после двух слэшей также является комментарием
\begin{verbatim}
// Место для комментария
// Комментарий может занимать только одну строку
\end{verbatim}




\subsection{Типы данных}
\begin{enumerate}
\item real числа с плавающей запятой \\
 $±1.17549435082 * 10^{-38} .. ±6.80564774407 * 10^{38}$

\item integer целые -32768 .. 32767

\item char символьный
\item boolean логический
\item перечисления
\end{enumerate}
пример:
\begin{lstlisting}

var           // объявления переменных
  r: Real;    // переменная вещественного типа
  i: Integer; // переменная целого типа
  c: Char;    // переменная - символ
  b: Boolean; // логическая переменная
  s: String;  // переменная строки
  t: Text;    // переменная для обьявления текстового файла

  e: (apple, banana, orange, lemon); // перечисление
  x: 1..10;    // переменная типа - перечисления
  y: 'a'..'z'; // переменная типа - перечисления

  set1: set of 1..10; // множество
  set2: set of 'a'..'z'; // множество

  r = record  // определение записи
        x: integer;
        y: char;
  end;
  f = Text;  // определение файла

\end{lstlisting}


\subsection{Блок (Составной оператор)}
Блок используется если можно использовать только один оператор, а
хочется несколько. Блок ограничивается ключевыми словами begin и end.

Например if (a>b) then оп1 else оп2;

вместо оп1 (или оп2) может быть только один оператор но часто нужно
выполнить несколько.

\begin{verbatim}
if a > b then begin
        оп3;
        оп4;
        оп5;
    end;
else
    оп2;
\end{verbatim}




\subsection{Операторы управления выполнением программы}
\begin{verbatim}
if a > b then  // условный оператор
  writeln('Условие выполнилось')
else           // иначе - секция может отсутствовать
  writeln('Условие не выполнилось');

case i of  // условный оператор множественного выбора
  0: write('ноль');
  1: write('один');
  2: write('два')
  else write('неизвестное число') // иначе - секция
                              // может отсутствовать
end;       // окончание case
           // один из случаев когда нет begin но есть end
\end{verbatim}

Для множественных условий предпочтительно использовать оператор case
(потому что компилятор в большинстве случаев создаст более оптимальный
код).

операторы сравнивнения
\begin{verbatim}
<  меньше
>  больше
<= меньше или равно
>= больше или равно
=  равно
<> неравно
\end{verbatim}

логические операторы
\begin{verbatim}
or  или
and и
not не
\end{verbatim}


{\tiny
  нежелательно делать так
\begin{verbatim}
   if (b=5) then ...
\end{verbatim}
  лучше так (на 3 курсе вам скажут что паскаль
  не нужен и обучат C, а привычки останутся)
\begin{verbatim}
   if (5=b) then ...
\end{verbatim}
  
  
  совсем неправильно делать так
\begin{verbatim}
program abc5;
var 
   a,b : real;
begin
   a:=7.0;
   b:=1.0-((1.0/3.0)*(a-1.0)/2.0); 
   if (b=0.0) then
      writeln('zero')
   else
      writeln ('no zero');
   writeln(b);
end.
\end{verbatim}

  более правильный вариант
\begin{verbatim}
program abc5;
const epsilon : real = 1.0e-10;
var 
   a,b : real;
begin
   a:=7.0;
   b:=1.0-((1.0/3.0)*(a-1.0)/2.0); 
   if (b < epsilon) then
      writeln('zero')
   else
      writeln ('no zero');
   writeln(b);
end.
\end{verbatim}

  в общем случае проверить равенство двух чисел (a,b) с плавающей
  запятой можно так, причём epsilon нужно выбирать исходя из числа
  разрядов, а также сложности и количества выполняемых действий
  (например r1 имеет погрешность $\pm1$\textohm,  r2 $\pm2$\textohm, посчитайте
  погрешность сопротивления при соединении резисторов параллельно по двум формулам
  $r=\frac{1}{\frac{1}{r_1} + \frac{1}{r_2}}$ и $r=\frac{r_1r_2}{r_1 + r_2}$)

\begin{verbatim}
   if (abs(a-b) < epsilon) then ...
   \\ или если хочется странного
   if (abs(a-b) < epsilon * (abs(a)+abs(b))) then ...
\end{verbatim}
}

\subsection{Циклы}
\begin{verbatim}
   n:=5;
   
   a:=1;
   while (a < n+1) do begin // цикл с предусловием
      writeln('a=', a);
      a := a+1;
   end;
   
   
   
   for b := 1 to 5 do begin // итерационный цикл
      writeln('b=', b);
      // внутри цикла for нельзя менять счётчик (b)
   end;
   // пользоваться счётчиком (b) после цикла некорректно
   
   
   c:=1;
   repeat // цикл с постусловием
      writeln('c=', c);
      c := c + 1;
   until (c > 5);
\end{verbatim}
В результате работы на экран будут выведены числа 1,2,3,4,5 в столбик.

\subsection{Процедуры и функции}
Процедуры отличаются от функций тем, что функции возвращают какое-либо
значение, а процедуры — нет.

\begin{verbatim}
program abc5;

var i : integer;

function next(k: integer): integer;
begin
    next := k + 1
end;
 
begin
  i := 1;
  writeln(next(i));
end.
\end{verbatim}

\subsection{Множества}
\begin{verbatim}
var { секция объявления переменных }
   d : set of char;
begin  { начало блока }
   d:=['a','b']; 
   i:=7;
   if i in [5..10] then writeln('принадлежит множеству');
\end{verbatim}


\section{Отчёт по программе}
\begin{enumerate}
\item Задание, а также описать задание как вы его поняли
\item Словесно-формульный алгоритм. Описать как работает алгоритм и
  рассмотреть сложные моменты
\item Блок-схема. Агромадный рисунок с кружочками, стрелочками и
  многоугольниками
\item Программа. Можно оставить пункт пустым: продемонстрировать
  исходный код программы
\item Руководство пользователя. Что нужно вводить и как получить
  результат.
\item Проверка. Если в программе вычисляется y:=sqrt(1/x) нужно
  проверить как работает программа при x=0.0; x=-9.0 и обычных числах например x=25.0
\item Улучшения. Большинство программ можно улучшить; нужно
  описать эти изменения, например:

  в программе присутствует ввод целого числа,
  но пользователь может ввести
  <<пять>>,
  << 5>> (пробел 5 [вообще то так можно делать]),
  <<=5>>,
  <<5O>> (буква O очень похожа на цифру 0),
  <<5,4>> (вместо 5.4 если спрашивают число с плавающей запятой).
  Всё это можно исправить если создать функцию, например <<readint>>,
  которая будет запрашивать ввод данных, предварительно обрабатывать
  их (например с помощью val), а в случае неправильного числа
  запрашивать ввод повторно
\item Лицензия. Указать название лицензии (в случае EULA привести
  полный текст лицензии)

\begin{enumerate}
\item BSD (делайте с программой что хотите: копируйте,
    изменяйте, распространяйте, продавайте)
\item GPL (делайте с программой что хотите:копируйте,
    изменяйте, распространяйте, продавайте. Но оставьте
    первоначального автора и лицензию GPL)
\item EULA (лицензионное соглашение с конечным пользователем) -
  договор между владельцем (автором) компьютерной программы и
 \sout{рабом} пользователем её копии. 
  {\tiny Студенту желающему сдать работу, и выбравшему в качестве
    лицензии EULA, требуется написать конечное соглашения пользователя
    в котором для примера, но не для бездумного копирования,
    используется в качестве основы, следующее описание, в котором
    описываются ограничения включающие, но не ограничивающиеся,
    запрещением просмотра исходного кода (только под NDA - соглашение
    о неразглашении), запрещение распространения, запрещение
    несанкционированного и несогласованного с высшим руководством
    запуска программы, запрещение продажи без покупки дистрибьюторских
    прав, банальные зонды и прочие соглашения почти не нарушающие
    конституцию и права человека, если будет доказано что пользователь
    действительно и неоспоримо на момент договора и в течении всего времени на
    которое распространяется действие договора время, являлся
    человеком, причём без возможности получения прямой либо косвенной выгоды
    в том числе либо материально либо нематериальной выгоды включая
    использование данного соглашения без изменения его сути и содержания,
    ограничиваясь только 10 (десятью) страницами мелкого, трудно
    читаемого текста.}
  {\tiny }
\end{enumerate}




\end{enumerate}


\newpage
%\section{литература}
\addcontentsline{toc}{chapter}{Список литературы}

\begin{thebibliography}{9}
\bibitem{alt.free.pascal}
  Е.Р.~Алексеев, О.В.~Чеснокова, Т.В.~Кучер\\
  \href{http://www.altlinux.org/Books:FreePascal}{Free Pascal и Lazarus: Учебник по программированию} \\
  Библиотека ALT Linux
  
\bibitem{elettroshop.quick.ref.pascal}
  \href{http://www.elettroshop.com/files/prodotti/download/mikroelektronica/pascal_syntax.pdf}{Quick Reference Guide for Pascal language} \\
  mikroElektronika SOFTWARE AND HARDWARE SOLUTIONS FOR THE EMBEDDED WORLD

\bibitem{ru.wikipedia.pascal}
  \href{http://ru.wikipedia.org/wiki/Pascal}{Паскаль (язык программирования)} \\
  Материал из Википедии — свободной энциклопедии 

\bibitem{zed.coding.rules}
  Жиганов Е.Д.  \href{http://zed.karelia.ru/go.to/for.students/coding.rules/rules}{/Студентам/Оформление программ/Правила} \\
  Как НУЖНО оформлять исходные тексты программ
  
\end{thebibliography}


\end{document}
