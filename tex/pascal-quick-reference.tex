%%% Local Variables: 
%%% mode: xelatex
%%% TeX-master: t
%%% End: 
 
%\documentclass[a4paper,10pt]{scrreprt}
%\documentclass[a4paper,10pt]{article}
%\documentclass[a4paper,10pt]{report}

\documentclass[unicode, 12pt, a4paper,oneside,fleqn]{article}
\usepackage[cm-default]{fontspec}
\defaultfontfeatures{Mapping=tex-text}    %% устанавливаем поведение шрифтов по умолчанию
\usepackage{polyglossia}    %% подключаем пакет многоязыкой верстки
%\setdefaultlanguage{russian}    %% установка языка по умолчанию
\setdefaultlanguage{english}    %% установка языка по умолчанию
%\setotherlanguages{english}
\setmainfont{Old Standard}      %% зададим основной шрифт документа
%\setmainfont{DejaVu Sans Mono}
\setmonofont{DejaVu Sans Mono}

\usepackage{mathtext}               % если нужны русские буквы в формулах
%\usepackage{ucs}
%\usepackage[utf8x]{inputenc}       % Кодировка входного документа;
                                   % при необходимости, вместо cp1251
                                   % можно указать cp866 (Alt-кодировка
                                   % DOS) или koi8-r.

\usepackage{textcomp}              % типографские значки

% \usepackage[T2A]{fontenc}           % Кодировка для шрифтов LH
%\usepackage{indentfirst}    % неизвестно
%\usepackage{cmap}           % неизвестно
%\usepackage[english,russian]{babel} % Включение русификации, русских и
                                    % английских стилей и переносов

%\renewcommand{\rmdefault}{ost_____} % add new font  Old_Standard
%\renewcommand{\sfdefault}{ost_____}
%\renewcommand{\ttdefault}{osti____}

\usepackage{graphics}
\usepackage{pgf}
\usepackage{wrapfig}
\usepackage{multicol}
\usepackage{multirow}
\usepackage{tabularx}
%\usepackage{fullpage}
%\usepackage{amsmath} % для спец знаков в формулах
%\usepackage{amssymb} % для спец знаков в формулах
\usepackage{topcapt} % подписи к таблицам
\usepackage{dcolumn} % выравнивание чисел
\usepackage{ulem} % подчёркивание

\hyphenpenalty=10000 % запретить переносы
%\tolerance-1 
\pretolerance10000 


\usepackage[colorlinks=true]{hyperref} % url hyperlink

\usepackage{makeidx} % индекс

%\newfontfamily\cyrillicfont{DejaVu Sans Mono}
%\newfontfamily{\cyrillicfont}{Liberation Mono}
\usepackage{verbatim} % печать неформатированного текста



\usepackage{listings} % печать исходного кода
%\lstloadlanguages{lisp}
\lstset{
  language=Pascal,
%  extendedchars=true, %Чтобы русские буквы в комментариях были 
  texcl,  %Чтобы русские буквы в комментариях были не слипшимися
  escapechar=|,
  frame=single,
  commentstyle=\itshape,
  inputencoding=utf8x,
  stringstyle=\bfseries
}



% вращение 
\usepackage{lscape}     % for %\begin{landscape} ...   %\end{landscape}
\usepackage{rotating}   % for sideways and \rotatebox{-90}{}




\author{Р.В.~Приходченко}
\title{Краткое руководство по языку программирования Pascal}
\frenchspacing


\makeindex

\begin{document}


\newcolumntype{Y}{>{\centering\arraybackslash}X}
%\newcolumntype{R}{>{\raggedleft\arraybackslash}X}


\maketitle
\tableofcontents


\begin{table}[ht]
  \begin{tabular}{cc}
    \includegraphics[width=2cm]{CC_BY-SA_88x31.png} &
    \shortstack{руководство распространяется в соответствии с
      условиями\\
      \href{http://creativecommons.org/licenses/by-sa/3.0/}{Attribution-ShareAlike} \\
      (Атрибуция — С сохранением условий) CC BY-SA \\
      Копирование и распространение приветствуется.}
  \end{tabular}
\end{table}

\section{Введение}
\subsection{Приблизительная последовательность действий при написании программы}
все действия этого параграфа происходят в терминале (консоль) кроме
пункта 4; пункты нужно выполнять последовательно 
\begin{enumerate}
\item создать каталог. Это необходимо сделать только один раз перед
  началом нового проекта (программы). Каждый проект хранится в
  отдельном каталоге. Каталог проекта будет содержать исходный код
  программы, исполняемые файлы, руководство пользователя и другие
  файлы необходимые для работы программы.\\
  пример:\\
  21119 - номер группы\\
  petroff - фамилия\\
  projabc3 - название проекта
\begin{verbatim}
mkdir -p 21119/petroff/projabc3
\end{verbatim}
  
\item перейти в каталог проекта
\begin{verbatim}
cd 21119/petroff/projabc3
\end{verbatim}

\item запустить любимый текстовый редактор (например emacs или gvim)\\
  abc3.pas - название программы\\
  \verb!&! (амперсанд) - интерпретатор (bash) не дожидается завершения
  команды, выполнение программы (emacs) происходит в фоновом режиме
  (в терминале можно вводить команды не останавливая emacs)
\begin{verbatim}
emacs abc3.pas &
\end{verbatim}

\item в текстовом редакторе (emacs) написать хорошую, правильную
  программу

\item компиляция программы\\
  abc3.pas - название программы
\begin{verbatim}
fpc abc3.pas
\end{verbatim}
  Однако лучше использовать гламурную компиляцию; для этого нужно в терминале
  ввести команду (не забудьте написать команду в одну строчку, а также
  поменять типографские кавычки на одинарные кавычки)
 {\scriptsize
\begin{verbatim}
function fpcc() { fpc "$1" 2>&1 | grep -Ei --color 'error|fatal|warning|note|'; }
\end{verbatim}
}
 
и запускать
\begin{verbatim}
fpcc abc3.pas
\end{verbatim}

% %%%%%%%%%%%%%%%%%%%%%%%%%%%%%%%%%%%%%%%%%%%%%%%%%%%%%%%%%%%%%%%%%%%%%%%%%%%%%%%%%%%%%%%%%%%%%%%%%%%%%%%%%%%%%%%%%%%%%%%%%%%%%%%%%%%%%%%%%%%%%%%%%%%%%%%%%%%%%%%%%%%%%%%%%%%%%%%%%%%%%%%%%%%%%%%%%%%%%%%%%%%%%%%%%%%%%%%%%%%%%%%%%%%%%%%%%%%%%%%%%%%%%%%%%%%%%%%%%%%%%%%%%%%%%%%%%%%%%%%%%%%%%%%%%%%%%%%%%%%%%%%%%%%%%%%%%%%%%%%%%%%%%%%%%%%%%%%%%%%%%%%%%%%%%%%%%%%%%%%%%%%%%%%%%%%%%%%%%%%%%%%%%%%%%%%%%%%%%%%%%%%%%%%%%%%%%%%%%%%%%%%%%%%%%%%%%%%%%%%%%%%%%%%%%%%%%%%%%%%%%%%%%%%%%%%%%%%%%%%%%%%%%%%%%%%%%%%%%%%%%%%%%%%%%%%%%%%%%%%%%%%%%%%%%%%%%%%%%%%%%%%%%%%%%%%%%%%%%%%%%%%%%%%%%%%%%%%%%%%%%%%%%%%%%%%%%%%%%%%%%%%%%%%%%%%%%%%%%%%%%%%%%%%%%%%%%%%%%%%%%%%%%%%%%%%%%%%%%%%%%%%%%%%%%%%%%%%%%%%%%%%%%%%%%%%%%%%%%%%%%%%%%%%%%%%%%%%%%%%%%%%%%%%%%%%%%%%%%%%%%%%%%%%%%%%%%%%%%%%%%%%%%%%%%%%%%%%%%%%%%%%%%%%%%%%%%%%%%%%%%%%%%%%%%%%%%%%%%%%%%%%%%%%%%%%%%%%%%%%%%%%%%%%%%%%%%%%%%%%%%%%%%%%%%%%%%%%%%%%%%%%%%%%%%%%%%%%%%%%%%%%%%%%%%%%%%%%%%%%%%%%%%%%%%%%%%%%%%%%%%%%%%%%%%%%%%%%%%%%%%%%%%%%%%%%%%%%%%%%%%%%%%%%%%%%%%%%%%%%%%%%%%%%%%%%%%%%%%%%%%%%%%%%%%%%%%%%%%%%%%%%%%%%%%%%%%%%%%%%%%%%%%%%%%%%%%%%%%%%%%%%%%%%%%%%%%%%%%%%%%%%%%%%%%%%%%%%%%%%%%%%%%%%%%%%%%%%%%%%%%%%%%%%%%%%%%%%%%%%%%%%%%%%%%%%%%%%%%%%%%%%%%%%%%%%%%%%%%%%%%%%%%%%%%%%%%%%%%%%%%%%%%%%%%%%%%%%%%%%%%%%%%%%%%%%%%%%%%%%%%%%%%%%%%%%%%%%%%%%%%%%%%%%%%%%%%%%%%%%%%%%%%%%%%%%%%%%%%%%%%%%%%%%%%%%%%%%%%%%%%%%%%%%%%%%%%%%%%%%%%%%%%%%%%%%%%%%%%%%%%%%%%%%%%%%%%%%%%%%%%%%%%%%%%%%%%%%%%%%%%%%%%%%%%%%%%%%%%%%%%%%%%%%%%%%%%%%%%%%%%%%%%%%%%%%%%%%%%%%%%%%%%%%%%%%%%%%%%%%%
%
% http://www.linux.org.ru/forum/development/4184158
% http://creativecommons.org/licenses/
% http://legroom.net/2009/08/18/bash-shell-aliases-and-functions
%
% %%%%%%%%%%%%%%%%%%%%%%%%%%%%%%%%%%%%%%%%%%%%%%%%%%%%%%%%%%%%%%%%%%%%%%%%%%%%%%%%%%%%%%%%%%%%%%%%%%%%%%%%%%%%%%%%%%%%%%%%%%%%%%%%%%%%%%%%%%%%%%%%%%%%%%%%%%%%%%%%%%%%%%%%%%%%%%%%%%%%%%%%%%%%%%%%%%%%%%%%%%%%%%%%%%%%%%%%%%%%%%%%%%%%%%%%%%%%%%%%%%%%%%%%%%%%%%%%%%%%%%%%%%%%%%%%%%%%%%%%%%%%%%%%%%%%%%%%%%%%%%%%%%%%%%%%%%%%%%%%%%%%%%%%%%%%%%%%%%%%%%%%%%%%%%%%%%%%%%%%%%%%%%%%%%%%%%%%%%%%%%%%%%%%%%%%%%%%%%%%%%%%%%%%%%%%%%%%%%%%%%%%%%%%%%%%%%%%%%%%%%%%%%%%%%%%%%%%%%%%%%%%%%%%%%%%%%%%%%%%%%%%%%%%%%%%%%%%%%%%%%%%%%%%%%%%%%%%%%%%%%%%%%%%%%%%%%%%%%%%%%%%%%%%%%%%%%%%%%%%%%%%%%%%%%%%%%%%%%%%%%%%%%%%%%%%%%%%%%%%%%%%%%%%%%%%%%%%%%%%%%%%%%%%%%%%%%%%%%%%%%%%%%%%%%%%%%%%%%%%%%%%%%%%%%%%%%%%%%%%%%%%%%%%%%%%%%%%%%%%%%%%%%%%%%%%%%%%%%%%%%%%%%%%%%%%%%%%%%%%%%%%%%%%%%%%%%%%%%%%%%%%%%%%%%%%%%%%%%%%%%%%%%%%%%%%%%%%%%%%%%%%%%%%%%%%%%%%%%%%%%%%%%%%%%%%%%%%%%%%%%%%%%%%%%%%%%%%%%%%%%%%%%%%%%%%%%%%%%%%%%%%%%%%%%%%%%%%%%%%%%%%%%%%%%%%%%%%%%%%%%%%%%%%%%%%%%%%%%%%%%%%%%%%%%%%%%%%%%%%%%%%%%%%%%%%%%%%%%%%%%%%%%%%%%%%%%%%%%%%%%%%%%%%%%%%%%%%%%%%%%%%%%%%%%%%%%%%%%%%%%%%%%%%%%%%%%%%%%%%%%%%%%%%%%%%%%%%%%%%%%%%%%%%%%%%%%%%%%%%%%%%%%%%%%%%%%%%%%%%%%%%%%%%%%%%%%%%%%%%%%%%%%%%%%%%%%%%%%%%%%%%%%%%%%%%%%%%%%%%%%%%%%%%%%%%%%%%%%%%%%%%%%%%%%%%%%%%%%%%%%%%%%%%%%%%%%%%%%%%%%%%%%%%%%%%%%%%%%%%%%%%%%%%%%%%%%%%%%%%%%%%%%%%%%%%%%%%%%%%%%%%%%%%%%%%%%%%%%%%%%%%%%%%%%%%%%%%%%%%%%%%%%%%%%%%%%%%%%%%%%%%%%%%%%%%%%%%%%%%%%%%%%%%%%%%%%%%%%%%%%%%%%%%%%%%%%%%%%%%%%%%%%%%%%%%%%%%%%%%%%%%%%%%%%%%%%%%%%%%%%%%%%%%%%%%%%%%%%%%%%%%%%%%%%%%%%%%%%%%


{\scriptsize

или можно создать файл \verb!~/bin/fpcc.sh! с таким содержимым:
\begin{verbatim}
#!/bin/sh
fpc $1 2>&1 | grep -Ei --color 'error|fatal|warning|note|'
\end{verbatim}
и запускать \verb!~/bin/fpcc.sh abc2.pas!
}

\item внимательно прочитать сообщения компилятора; при наличии ошибок
  или предупреждений перейти к пункту 4 (о сообщениях компилятора см. ниже)
  
\item запустить программу\\
  abc3 - название исполняемого файла (без расширения <<.pas>>)
\begin{verbatim}
./abc3
\end{verbatim}
  
\item если программа получилась негодной, перейти к пункту 4

\item если для демонстрации программы необходимо построить график то
  получаем текстовый файл с несколькими колонками разделёнными
  запятыми (без лишних сообщений), например так:
{\tiny
\begin{verbatim}
program abc5;
uses math;
const
   h: real = 1.0e-1;
var 
   a,b,c : real;

begin
   a:=0.0;
   b:=5.0;
   c:=a;
   repeat
      writeln(c, ', ', sin(c));
      c := c + h;
   until (c>b);
   
end.
\end{verbatim}
}
\begin{verbatim}
./abc3 > data.txt
\end{verbatim}

  
  для построения графика можно воспользоваться программой R или
  gnuplot (в них можно строить даже трёхмерные поверхности)
  \begin{enumerate}
  \item R
    
    запускаем в терминале R
\begin{verbatim}
gr <- read.table("data.txt", sep=",", head=FALSE)
plot(gr, type="l")
\end{verbatim}
  \item gnuplot
    
    запускаем в терминале gnuplot
\begin{verbatim}
plot "data.txt" with line
\end{verbatim}
  \end{enumerate}
\end{enumerate}

выход <<Ctrl d>>




\subsection{Сообщения компилятора}

Компилятор показывает сообщения об ошибках с номером строки и номером
символа в круглых скобках.\\
Например (6,4) - ошибка в строке 6, номер символа 4.\\
Однако если отсутствует \verb!;! (точка с запятой) в конце оператора
то компилятор укажет на следующую строку (пропущенную точку с запятой,
скорее всего, нужно добавить строкой выше). Если вы воспользовались
гламурной компиляцией (смотри выше) то ключевые слова будут подсвечены
цветом. 
 
Если в процессе компиляции появляются сообщения со словами <<error>>
или <<fatal>>, то в программе присутствует ошибка, которую необходимо
исправить. Например ошибки синтаксиса и операции с различными типами:
{\tiny
\begin{verbatim}
abc3.pas(6,4) Fatal: Syntax error, "." expected but ";" found
abc3.pas(7,4) Error: Incompatible types: got "String" expected "Real"
abc3.pas(10) Fatal: There were 1 errors compiling module, stopping
Fatal: Compilation aborted
\end{verbatim}
}

Если в процессе компиляции появляются сообщения со словами
<<warning>> или <<note>>, то в программе присутствует недостаток,
котоый желательно исправить. Например неиспользуемая переменная и
неинициализированная переменная (объявили переменную, в неё ничего не
записали, попытались вывести её значение на экран):
{\tiny
\begin{verbatim}
abc3.pas(3,7) Note: Local variable "c" not used
abc3.pas(10,16) Warning: Variable "b" does not seem to be initialized
\end{verbatim}
}

успешно откомпилированная программа должна содержать примерно такую строку:
\begin{verbatim}
10 lines compiled, 0.0 sec
\end{verbatim}


\section{Паскаль}
примерный вид программы

%\begin{lstlisting}
%  странно но несмотря на texcl русские буквы в комментариях слипаются
%\end{verbatim}
\begin{verbatim}
program abc3;     // название программы (начинается с буквы)

uses math;       // подключение модулей
                // (в данном случае для математических функций)

const             // список констант
   MAX : integer = 100;  

type
   mass : array [1..MAX] of integer;


var               // описание типов переменных
   a : integer;
   s : string;
   m : mass;

begin             // начало программы

   readln(s);     // программа
   m[2] := 7;     // программа
   m[MAX-8] := 3; // программа
   a := 5;        // программа
   writeln(s, a); // программа
   
end.              // конец программы (end с точкой)

\end{verbatim}



\subsection{Ключевые слова}
ключевые слова не допустимо использовать для названия переменных,
констант, процедур и функций.
список ключевых слов:

absolute, and, array, asm, begin, boolean, break, case, char, const, continue, div, do, downto, else, end, for, function, goto, if, implementation, in, interrupt, is, label, mod, not, or, org, otherwise, print, procedure, program, read, real, record, repeat, shl, shr, step, string, then, to, type, unit, until, uses, var, while, with, xor

\subsection{Комментарии}
текст заключённый между фигурными скобками - комментарии к программе 
Пример: 
\begin{verbatim}
{ Место для комментария
  Комментарий может занимать несколько строк }
\end{verbatim}

текст после двух слэшей также является комментарием
\begin{verbatim}
// Место для комментария
// Комментарий может занимать только одну строку
\end{verbatim}




\subsection{Типы данных}
\begin{enumerate}
\item real числа с плавающей запятой \\
 $±1.17549435082 * 10^{-38} .. ±6.80564774407 * 10^{38}$

\item integer целые -32768 .. 32767

\item char символьный
\item boolean логический
\item перечисления
\end{enumerate}
пример:
\begin{verbatim}

var        // объявления переменных
  r: Real;    // переменная вещественного типа
  i: Integer; // переменная целого типа
  c: Char;    // переменная-символ
  b: Boolean; // логическая переменная
  s: String;  // переменная строки
   t: Text;      // переменная для обьявления текстового файла

  e: (apple, pear, banana, orange, lemon); 
             // переменная типа-перечисления
  x: 1..10;  // переменная типа-перечисления
  y: 'a'..'z'; // переменная типа-перечисления

  set1: set of 1..10; // множество
  set2: set of 'a'..'z'; // множество

  r = record  // определение записи
        x: integer;
        y: char;
  end;
  f = Text;  // определение файла

\end{verbatim}


\subsection{Блок (Составной оператор)}
Блок используется если можно использовать только один оператор, а
хочется несколько. Блок ограничивается ключевыми словами begin и end. 

Например if (a>b) then оп1 else оп2;

вместо оп1 (или оп2) может быть только один оператор но часто нужно
выполнить несколько.
\begin{verbatim}
if a > b then begin
        оп3;
        оп4;
        оп5;
    end;
else
    оп2;
\end{verbatim}




\subsection{Операторы управления выполнением программы}
\begin{verbatim}
if a > b then  // условный оператор
  writeln('Условие выполнилось')
else           // иначе - секция может отсутствовать
  writeln('Условие не выполнилось');

case i of  // условный оператор множественного выбора
  0: write('ноль');
  1: write('один');
  2: write('два')
  else write('неизвестное число') // иначе - секция
                              // может отсутствовать
end;       // окончание case
           // один из случаев когда нет begin но есть end
\end{verbatim}

Для множественных условий предпочтительно использовать оператор case (потому что компилятор в большинстве случаев создаст более оптимальный код).

операторы сравнивнения
\begin{verbatim}
<  меньше
>  больше
<= меньше или равно
>= больше или равно
=  равно
<> неравно
\end{verbatim}

логические операторы
\begin{verbatim}
or  или
and и
not не
\end{verbatim}


{\tiny
нежелательно делать так
\begin{verbatim}
   if (b=5) then ...
\end{verbatim}
лучше так (на 3 курсе вам скажут что паскаль
  не нужен и обучат C, а привычки останутся)
\begin{verbatim}
   if (5=b) then ...
\end{verbatim}


совсем неправильно делать так
\begin{verbatim}
program abc5;
var 
   a,b : real;
begin
   a:=7.0;
   b:=1.0-((1.0/3.0)*(a-1.0)/2.0); 
   if (b=0.0) then
      writeln('zero')
   else
      writeln ('no zero');
   writeln(b);
end.
\end{verbatim}

более правильный вариант
\begin{verbatim}
program abc5;
const epsilon : real = 1.0e-10;
var 
   a,b : real;
begin
   a:=7.0;
   b:=1.0-((1.0/3.0)*(a-1.0)/2.0); 
   if (b < epsilon) then
      writeln('zero')
   else
      writeln ('no zero');
   writeln(b);
end.
\end{verbatim}

в общем случае проверить равенство двух чисел (a,b) с плавающей
запятой можно так, причём epsilon нужно выбирать исходя из числа
разрядов, а также сложности и количества выполняемых действий
(например r1 имеет погрешность $\pm1$\textohm,  r2 $\pm2$\textohm, посчитайте
погрешность сопротивления при соединении резисторов параллельно по двум формулам
$r=\frac{1}{\frac{1}{r_1} + \frac{1}{r_2}}$ и $r=\frac{r_1r_2}{r_1 + r_2}$)

\begin{verbatim}
   if (abs(a-b) < epsilon) then ...
   \\ или если хочется странного
   if (abs(a-b) < epsilon * (abs(a)+abs(b))) then ...
\end{verbatim}
}

\subsection{Циклы}
\begin{verbatim}
   n:=5;
   
   a:=1;
   while (a < n+1) do begin // цикл с предусловием
      writeln('a=', a);
      a := a+1;
   end;
   
   
   
   for b := 1 to 5 do begin // итерационный цикл
      writeln('b=', b);
      // внутри цикла for нельзя менять счётчик (b)
   end;
   // пользоваться счётчиком (b) после цикла некорректно
   
   
   c:=1;
   repeat // цикл с постусловием
      writeln('c=', c);
      c := c + 1;
   until (c > 5);
\end{verbatim}
В результате работы на экран будут выведены числа 1,2,3,4,5 в столбик.

\subsection{Процедуры и функции}
Процедуры отличаются от функций тем, что функции возвращают какое-либо
значение, а процедуры — нет.

\begin{verbatim}
program abc5;

var i : integer;

function next(k: integer): integer;
begin
    next := k + 1
end;
 
begin
  i := 1;
  writeln(next(i));
end.
\end{verbatim}

\subsection{Множества}
\begin{verbatim}
var { секция объявления переменных }
   d : set of char;
begin  { начало блока }
   d:=['a','b']; 
   i:=7;
   if i in [5..10] then writeln('принадлежит множеству');
\end{verbatim}


\section{Отчёт по программе}
\begin{enumerate}
\item Задание. Описать задание как вы его поняли
\item Словесно-формульный алгоритм. Описать как работает алгоритм и
  рассмотреть сложные моменты
\item Блок-схема. Агромадный рисунок с кружочками, стрелочками и
  многоугольниками
\item Программа. Можно оставить пункт пустым: продемонстрировать
  работающую программу
\item Руководство пользователя. Что нужно вводить и как получить результат
\item Проверка. Если в программе вычисляется y:=sqrt(1/x) нужно
  проверить как работает программа при x=0.0; x=-9.0 и обычных числах например x=25.0
\item Улучшения. Большинство программ можно улучшить; нужно
  описать эти изменения например в программе присутствует ввод целого числа,
  но пользователь может ввести
  <<пять>>,
  << 5>> (пробел 5 [вообще то так можно делать]),
  <<=5>>,
  <<5O>> (буква O очень похожа на цифру 0),
  <<5,4>> (вместо 5.4 если спрашивают число с плавающей запятой).
  Всё это можно исправить если создать функцию, например <<readint>>,
  которая будет запрашивать ввод данных, предварительно обрабатывать
  их (например с помощью val), а в случае неправильного числа
  запрашивать ввод повторно
\item Лицензия. Указать название лицензии (в случае EULA привести
  полный текст лицензии)

\begin{enumerate}
\item BSD (делайте с программой что хотите: копируйте,
    изменяйте, распространяйте, продавайте)
\item GPL (делайте с программой что хотите:копируйте,
    изменяйте, распространяйте, продавайте. Но оставьте
    первоначального автора и лицензию GPL)
\item EULA (лицензионное соглашение с конечным пользователем) -
  договор между владельцем (автором) компьютерной программы и
 \sout{рабом} пользователем её копии. 
  {\tiny Студенту желающему сдать работу, и выбравшему в качестве
    лицензии EULA, требуется написать конечное соглашения пользователя
    в котором для примера, но не для бездумного копирования,
    используется в качестве основы, следующее описание, в котором
    описываются ограничения включающие, но не ограничивающиеся,
    запрещением просмотра исходного кода (только под NDA - соглашение
    о неразглашении), запрещение распространения, запрещение
    несанкционированного и несогласованного с высшим руководством
    запуска программы, запрещение продажи без покупки дистрибьюторских
    прав, банальные зонды и прочие соглашения почти не нарушающие
    конституцию и права человека, если будет доказано что пользователь
    действительно и неоспоримо на момент договора и в течении всего времени на
    которое распространяется действие договора время, являлся
    человеком, причём без возможности получения прямой либо косвенной выгоды
    в том числе либо материально либо нематериальной выгоды включая
    использование данного соглашения без изменения его сути и содержания,
    ограничиваясь только 10 (десятью) страницами мелкого, трудно
    читаемого текста.}
\end{enumerate}




{\tiny }
\end{enumerate}


\section{литература}


\begin{enumerate}
\item
  Е.Р.~Алексеев, О.В.~Чеснокова, Т.В.~Кучер\\
  \href{http://www.altlinux.org/Books:FreePascal}{Free Pascal и Lazarus: Учебник по программированию} \\
  Библиотека ALT Linux
  
\item
  \href{http://www.elettroshop.com/files/prodotti/download/mikroelektronica/pascal_syntax.pdf}{Quick Reference Guide for Pascal language} \\
  mikroElektronika SOFTWARE AND HARDWARE SOLUTIONS FOR THE EMBEDDED WORLD
  
\item
  \href{http://ru.wikipedia.org/wiki/Pascal}{Паскаль (язык программирования)} \\
  Материал из Википедии — свободной энциклопедии 
\end{enumerate}


\end{document}
