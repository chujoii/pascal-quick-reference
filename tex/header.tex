%\documentclass[a4paper,10pt]{scrreprt}
%\documentclass[a4paper,10pt]{article}
%\documentclass[a4paper,10pt]{report}

\documentclass[unicode, 12pt, a4paper,oneside,fleqn]{article}
\usepackage[cm-default]{fontspec}
\defaultfontfeatures{Mapping=tex-text}    %% устанавливаем поведение шрифтов по умолчанию
\usepackage{polyglossia}    %% подключаем пакет многоязыкой верстки
%\setdefaultlanguage{russian}    %% установка языка по умолчанию
\setdefaultlanguage{english}    %% установка языка по умолчанию
%\setotherlanguages{english}
\setmainfont{Old Standard TT}      %% зададим основной шрифт документа
%\setmainfont{DejaVu Sans Mono}
\setmonofont{DejaVu Sans Mono}

\usepackage{mathtext}               % если нужны русские буквы в формулах
%\usepackage{ucs}
%\usepackage[utf8x]{inputenc}       % Кодировка входного документа;
                                   % при необходимости, вместо cp1251
                                   % можно указать cp866 (Alt-кодировка
                                   % DOS) или koi8-r.

\usepackage{textcomp}              % типографские значки

% \usepackage[T2A]{fontenc}           % Кодировка для шрифтов LH
%\usepackage{indentfirst}    % неизвестно
%\usepackage{cmap}           % неизвестно
%\usepackage[english,russian]{babel} % Включение русификации, русских и
                                    % английских стилей и переносов

%\renewcommand{\rmdefault}{ost_____} % add new font  Old_Standard
%\renewcommand{\sfdefault}{ost_____}
%\renewcommand{\ttdefault}{osti____}

\usepackage{graphics}
\usepackage{pgf}
\usepackage{wrapfig}
\usepackage{multicol}
\usepackage{multirow}
\usepackage{tabularx}
%\usepackage{fullpage}
%\usepackage{amsmath} % для спец знаков в формулах
%\usepackage{amssymb} % для спец знаков в формулах
\usepackage{topcapt} % подписи к таблицам
\usepackage{dcolumn} % выравнивание чисел
\usepackage{ulem} % подчёркивание

\hyphenpenalty=10000 % запретить переносы
%\tolerance-1 
\pretolerance10000 


\usepackage[colorlinks=true]{hyperref} % url hyperlink

\usepackage{makeidx} % индекс

%\newfontfamily\cyrillicfont{DejaVu Sans Mono}
%\newfontfamily{\cyrillicfont}{Liberation Mono}
\usepackage{verbatim} % печать неформатированного текста
%\usepackage{verbatimbox} % печать неформатированного текста в рамке
\usepackage{fancyvrb}


\usepackage{listings} % печать исходного кода
%\lstloadlanguages{lisp}
\lstset{
  language=Pascal,
  %basicstyle=\tiny, %or \small or \footnotesize etc.
  extendedchars=\true, %Чтобы русские буквы в комментариях были 
  %texcl,  %Чтобы русские буквы в комментариях были не слипшимися
  keepspaces = true, %Чтобы русские буквы в комментариях были не слипшимися это работает!!!!!!!
  escapechar=|,
  frame=single,
  commentstyle=\itshape,
  inputencoding=utf8x,
  stringstyle=\bfseries
}



% вращение 
\usepackage{lscape}     % for %\begin{landscape} ...   %\end{landscape}
\usepackage{rotating}   % for sideways and \rotatebox{-90}{}



