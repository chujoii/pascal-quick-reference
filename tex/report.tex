%\documentclass[a4paper,10pt]{scrreprt}
%\documentclass[a4paper,10pt]{article}
%\documentclass[a4paper,10pt]{report}

\documentclass[unicode, 12pt, a4paper,oneside,fleqn]{article}
\usepackage[cm-default]{fontspec}
\defaultfontfeatures{Mapping=tex-text}    %% устанавливаем поведение шрифтов по умолчанию
\usepackage{polyglossia}    %% подключаем пакет многоязыкой верстки
%\setdefaultlanguage{russian}    %% установка языка по умолчанию
\setdefaultlanguage{english}    %% установка языка по умолчанию
%\setotherlanguages{english}
\setmainfont{Old Standard}      %% зададим основной шрифт документа
%\setmainfont{DejaVu Sans Mono}
\setmonofont{DejaVu Sans Mono}

\usepackage{mathtext}               % если нужны русские буквы в формулах
%\usepackage{ucs}
%\usepackage[utf8x]{inputenc}       % Кодировка входного документа;
                                   % при необходимости, вместо cp1251
                                   % можно указать cp866 (Alt-кодировка
                                   % DOS) или koi8-r.

\usepackage{textcomp}              % типографские значки

% \usepackage[T2A]{fontenc}           % Кодировка для шрифтов LH
%\usepackage{indentfirst}    % неизвестно
%\usepackage{cmap}           % неизвестно
%\usepackage[english,russian]{babel} % Включение русификации, русских и
                                    % английских стилей и переносов

%\renewcommand{\rmdefault}{ost_____} % add new font  Old_Standard
%\renewcommand{\sfdefault}{ost_____}
%\renewcommand{\ttdefault}{osti____}

\usepackage{graphics}
\usepackage{pgf}
\usepackage{wrapfig}
\usepackage{multicol}
\usepackage{multirow}
\usepackage{tabularx}
%\usepackage{fullpage}
%\usepackage{amsmath} % для спец знаков в формулах
%\usepackage{amssymb} % для спец знаков в формулах
\usepackage{topcapt} % подписи к таблицам
\usepackage{dcolumn} % выравнивание чисел
\usepackage{ulem} % подчёркивание

\hyphenpenalty=10000 % запретить переносы
%\tolerance-1 
\pretolerance10000 


\usepackage[colorlinks=true]{hyperref} % url hyperlink

\usepackage{makeidx} % индекс

%\newfontfamily\cyrillicfont{DejaVu Sans Mono}
%\newfontfamily{\cyrillicfont}{Liberation Mono}
\usepackage{verbatim} % печать неформатированного текста
%\usepackage{verbatimbox} % печать неформатированного текста в рамке
\usepackage{fancyvrb}


\usepackage{listings} % печать исходного кода
%\lstloadlanguages{lisp}
\lstset{
  language=Pascal,
  %basicstyle=\tiny, %or \small or \footnotesize etc.
  extendedchars=\true, %Чтобы русские буквы в комментариях были 
  %texcl,  %Чтобы русские буквы в комментариях были не слипшимися
  keepspaces = true, %Чтобы русские буквы в комментариях были не слипшимися это работает!!!!!!!
  escapechar=|,
  frame=single,
  commentstyle=\itshape,
  inputencoding=utf8x,
  stringstyle=\bfseries
}



% вращение 
\usepackage{lscape}     % for %\begin{landscape} ...   %\end{landscape}
\usepackage{rotating}   % for sideways and \rotatebox{-90}{}



 % длинный заголовок файла

\author{Р.В.~Приходченко}

\title{Образец отчёта}

\date{2016 Апрель 01 (не забудьте указать правильную дату написания
  отчёта или вообще удалите эту строку, и тогда \TeX создаст её
  автоматически)}

\frenchspacing


\makeindex

\begin{document}

% меняем английские термины на русские
\renewcommand\bibname{СПИСОК ЛИТЕРАТУРЫ} 
\renewcommand\refname{\centering Список литературы}
\renewcommand\contentsname{\centering Содержание}


\newcolumntype{Y}{>{\centering\arraybackslash}X}
%\newcolumntype{R}{>{\raggedleft\arraybackslash}X}

% печатаем титульный лист
\maketitle

% печатаем оглавление
\tableofcontents







\section{Лицензия}

% <------ процент это комментарий,
% чтобы сделать текст видимым удалите символ процента в начале строки

% Указать название лицензии (в случае EULA привести полный текст лицензии)
% Вместо лицензии CC BY-SA для финального отчёта можно выбрать любую:

Руководство распространяется в соответствии с
    условиями \href{http://creativecommons.org/licenses/by-sa/3.0/}{Attribution-ShareAlike} 
    (Атрибуция — С сохранением условий) CC BY-SA %\includegraphics[width=2cm]{CC_BY-SA_88x31.png}
    Копирование и распространение приветствуется.

%BSD (делайте с программой что хотите: копируйте,
%    изменяйте, распространяйте, продавайте)

%GPL (делайте с программой что хотите:копируйте,
%    изменяйте, распространяйте, продавайте. Но оставьте
%    первоначального автора и лицензию GPL)

%EULA (лицензионное соглашение с конечным пользователем) -
%  договор между владельцем (автором) компьютерной программы и
% \sout{рабом} пользователем её копии. 
%  {\tiny Студенту желающему сдать работу, и выбравшему в качестве
%    лицензии EULA, требуется написать конечное соглашения пользователя
%    в котором для примера, но не для бездумного копирования,
%    используется в качестве основы, следующее описание, в котором
%    описываются ограничения включающие, но не ограничивающиеся,
%    запрещением просмотра исходного кода (только под NDA - соглашение
%    о неразглашении), запрещение распространения, запрещение
%    несанкционированного и несогласованного с высшим руководством
%    запуска программы, запрещение продажи без покупки дистрибьюторских
%    прав, банальные зонды и прочие соглашения почти не нарушающие
%    конституцию и права человека, если будет доказано что пользователь
%    действительно и неоспоримо на момент договора и в течении всего времени на
%    которое распространяется действие договора время, являлся
%    человеком, причём без возможности получения прямой либо косвенной выгоды
%    в том числе либо материально либо нематериальной выгоды включая
%    использование данного соглашения без изменения его сути и содержания,
%    ограничиваясь только 10 (десятью) страницами мелкого, трудно
%    читаемого текста.}
%  {\tiny }









\section{Задание}
Описать задание: как вы его поняли.







\section{Словесно-формульный алгоритм}
Рассмотреть как работает алгоритм, в особенности сложные моменты
алгоритма.

Если требуется список, то воспользуйтесь конструкцией:
\begin{enumerate} % начало списка
\item в начале необходимо накопить умных мыслей
\item открываем текстовый редактор
\item ...
\item PROFIT!
\end{enumerate} % конец списка







\section{Блок-схема}
Агромадный рисунок с кружочками, стрелочками и многоугольниками:

Если вы использовали программу Dia для создания блок-схемы, то после
сохранения в каталоге проекта (в формате .dia) экспортируйте схему в
Encapsulated PostScript используя шрифты Pango (*.eps), например
<<flowchart.eps>>. Также можно задать размер (ширину) картинки:
width=15cm

\includegraphics[width=15cm]{flowchart.eps}






\section{Программа}
Продемонстрировать исходный код программы

Пример демонстрации кусочка программы:
\begin{lstlisting}
  ...
  writeln ('цикл с предусловием:');
  i:=1;                    {начальное значение счётчика}
  while (i <= max_n) do begin {проверяем условие выхода}
  writeln ('i=', i);
  i := i + 1;         {увеличиваем счётчик на 1}
  end;
  writeln (); {перевод строки}
  ...
\end{lstlisting}


или отдельного файла (вместо <<test.pas>> подставить ваш файл, в
квадратных скобочках можно указать дополнительные параметры: толщина
рамки, тень под рамкой, номера строк, ...):
\lstinputlisting[]{test.pas}







\section{Руководство пользователя}
Что нужно вводить и как получить результат.

\begin{enumerate}
\item перейти в каталог проекта
  \begin{lstlisting}[language=bash]
    cd 21119/petroff/mysuperprogramm
  \end{lstlisting}
  
\item запустить программу
  \begin{lstlisting}[language=bash]
    ./mysuperprogramm
  \end{lstlisting}
  
\item программа попросит ввести произвольную строку, после чего
  пользователь вводит произвольную строку и нажимает ввод (Enter)
  \begin{lstlisting}[language=bash]
    введите произвольную строку = строка состоит из букв.
  \end{lstlisting}

\item программа обрабатывает введённые данные и выводит результат
  \begin{lstlisting}[language=bash]
    Это результат работы программы. Так- то!
  \end{lstlisting}
  
\item внимательно прочитать результат и при необходимости запустить
  программу снова
  
\end{enumerate}







\section{Проверка}
Если в программе используются маленькие формулы, например
$y = \sqrt{1/x} = \sqrt{\frac{1}{x}}$ нужно проверить, как работает
программа при $x=0.0$; $x=-9.0$ и обычных числах например x=25

Для больших формул можно воспользоваться конструкцией <<equation>>:
\begin{equation}
  E = m_1 \cdot c^2
\end{equation}







\section{Улучшения}
Большинство программ можно улучшить; нужно
описать эти изменения, например:

В программе присутствует ввод целого числа,
но пользователь может ввести:
\begin{itemize}
\item <<пять>>
\item << 5>> (пробел 5 [так можно делать])
\item <<=5>>
\item <<5O>> (буква O очень похожа на цифру 0)
\item <<5,4>> (вместо 5.4 если спрашивают число с плавающей запятой)
\end{itemize}

Всё это можно исправить, если создать функцию, например <<readint>>,
которая будет запрашивать ввод данных, предварительно обрабатывать их
(например с помощью val), а в случае неправильного числа запрашивать
ввод повторно.







\end{document}
